%==================================================================
% Ini adalah bab 2
% Silahkan edit sesuai kebutuhan, baik menambah atau mengurangi \section, \subsection
%==================================================================

\chapter[RENCANA KEGIATAN PRAKTIK INDUSTRI]{\\ RENCANA KEGIATAN PRAKTIK INDUSTRI}

\section{Jadwal Kegiatan}
Berikan rincian jadwal pelaksanaan praktik industri. Sebutkan tanggal mulai dan berakhirnya kegiatan, serta pembagian waktu untuk setiap tahap atau aktivitas yang direncanakan. Jadwal yang jelas akan membantu dalam perencanaan dan pelaksanaan yang efektif.

\section{Deskripsi Kegiatan}
Jelaskan secara detail setiap kegiatan yang akan dilakukan selama praktik industri. Sebutkan tujuan dari setiap kegiatan, serta bagaimana kegiatan tersebut berkontribusi terhadap pencapaian tujuan keseluruhan praktik industri. Deskripsi ini harus memberikan gambaran yang jelas tentang aktivitas sehari-hari yang akan dilakukan oleh peserta.

\subsection{Kegiatan 1}
Jelaskan secara detail setiap aktivitas pada kegiatan 1 yang akan dilakukan selama Praktik Industri.

\subsection{Kegiatan 2}
Jelaskan secara detail setiap aktivitas pada kegiatan 1 yang akan dilakukan selama Praktik Industri.

\subsection{Kegiatan 3}
Jelaskan secara detail setiap aktivitas pada kegiatan 1 yang akan dilakukan selama Praktik Industri.

\section{Lokasi dan Fasilitas}
Sebutkan lokasi di mana praktik industri akan dilaksanakan. Jelaskan fasilitas apa saja yang akan disediakan oleh pihak perusahaan atau institusi tempat praktik berlangsung, seperti alat dan bahan yang diperlukan, ruang kerja, dan dukungan teknis lainnya.

\subsection{Profil Perusahaan \perusahaan}
Berikan penjelasan terkait profil perusahaan yang digunakan untuk Praktik Industri.

\subsection{Peralatan yang digunakan}
Berikan penjelasan terkait peralatan pada perusahaan yang rencana akan digunakan untuk Praktik Industri.

\section{Konversi Mata Kuliah}
Berikan penjelasan terkait peluang mata kuliah yang dapat dijadikan wadah konversi dari Praktik Industri yang dilakukan.

\section{Evaluasi dan Pelaporan}
Deskripsikan bagaimana hasil dari praktik industri akan dievaluasi. Jelaskan bagaimana proses pelaporan hasil kegiatan akan dilakukan, termasuk format laporan dan tenggat waktu pengumpulannya.
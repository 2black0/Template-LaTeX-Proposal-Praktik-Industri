%==================================================================
% Ini adalah bab 3
% Silahkan edit sesuai kebutuhan, baik menambah atau mengurangi \section, \subsection
%==================================================================

\chapter[PESERTA DAN PEMBIMBING]{\\ PESERTA DAN PEMBIMBING}

\section{Peserta}
Jelaskan kriteria peserta yang akan mengikuti praktik industri, seperti latar belakang pendidikan, keterampilan yang dimiliki, dan jumlah peserta yang diharapkan. Deskripsi ini penting untuk memastikan bahwa peserta yang terlibat memiliki kualifikasi yang sesuai. Pada bagian ini juga harus menampilkan gambar kartu mahasiswa aktif.

\section{Pembimbing}
Sebutkan kriteria pembimbing yang akan mendampingi peserta selama praktik industri. Jelaskan peran dan tanggung jawab pembimbing, termasuk memberikan bimbingan, dukungan, dan evaluasi terhadap perkembangan peserta.

\section{Proses Pembimbingan}
Jelaskan mekanisme koordinasi dan pembimbingan antara peserta dan pembimbing. Sebutkan saluran komunikasi yang akan digunakan, seperti rapat rutin, email, atau platform komunikasi lainnya, untuk memastikan kelancaran pelaksanaan praktik industri.